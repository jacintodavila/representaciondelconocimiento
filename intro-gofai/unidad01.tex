\documentclass[10pt]{article}
%\usepackage{fleqn}
\usepackage{epsf}
\usepackage[dvips]{color}
\usepackage{aima2e-slides}
%\usepackage{lscape} 
%\usepackage{pdflscape}
\special{landscape}
%\usepackage[landscape]{geometry}

\begin{document}

%\begin{landscape}

\begin{huge}
\titleslide{Inteligencia Artificial}{Unidad 1}

\sf

%%%%%%%%%%%% Slide %%%%%%%%%%%%%%%%%%%%%%%%%%%%%%%%%%%%%%%%%%%%%%%%%%%
\heading{Contenido}

\blob Visi{\'o}n y misi{\'o}n del curso

\blob ?`Qu{\'e} es Inteligencia Artificial?

\blob La historia resumida

\blob ``El estado del arte"

%%%%%%%%%%%% Slide %%%%%%%%%%%%%%%%%%%%%%%%%%%%%%%%%%%%%%%%%%%%%%%%%%%
\heading{Administrivia}

Sitios Web: 
\verb|http://webdelprofesor.ula.ve/ingenieria/jacinto|\\
\verb|http://jacinto-davila.blogspot.com|\\

Libro: Russell and Norvig \u{Artificial Intelligence: A Modern Approach}\\

C{\'o}digos: los programas del libro en lisp, PROLOG, C++, Python
y Java est{\'a}n en\\

\verb|http://aima.cs.berkeley.edu/|\\
\verb|http://www.doc.ic.ac.uk/~rak/|\\


%%%%%%%%%%%% Slide %%%%%%%%%%%%%%%%%%%%%%%%%%%%%%%%%%%%%%%%%%%%%%%%%%%
\heading{Misi{\'o}n y Visi{\'o}n del curso}

\blob Visi{\'o}n: Un entorno social m{\'a}s inteligente, causa y
consecuencia del trabajo colectivo\\

\blob Misi{\'o}n: Proveer herramientas tecnol{\'o}gicas para ese
trabajo colectivo (que aprovechen y que conduzcan a m{\'a}s
inteligencia).

\blob Hip{\'o}tesis: La inteligencia artificial puede favorecer a
la inteligencia na\-tural.

\blob Contexto: Un mundo en postpandemia (?`Qu{\'e} tiene que ver el contexto?).

%%%%%%%%%%%% Slide %%%%%%%%%%%%%%%%%%%%%%%%%%%%%%%%%%%%%%%%%%%%%%%%%%%
\heading{?`Cu{\'a}les temas contiene este curso?}

\blob Agentes inteligentes ($\surd$)\\
\blob B{\'u}squeda ($\surd$) y juegos \\
\blob Sistemas basados en l{\'o}gica ($\surd$)\\
\blob Sistemas de planificaci{\'o}n ($\surd$)\\
\blob Incertidumbre, probabilidad y teoria de decisiones en IA($\surd$)\\
\blob Aprendizaje ($\surd$)\\
\blob Lenguaje\\
\blob Percepci{\'o}n\\
\blob Rob{\'o}tica\\
\blob El contacto con la filosof{\'\i}a\\


%%%%%%%%%%%% Slide %%%%%%%%%%%%%%%%%%%%%%%%%%%%%%%%%%%%%%%%%%%%%%%%%%%
\heading{?`Qu{\'e} es la Inteligencia Artificial?}

\vspace*{1in}

La tabla de Russell y Norvig:

\vspace*{1in}

\centerline{\input{tables/ai-is-table}}


%%%%%%%%%%%% Slide %%%%%%%%%%%%%%%%%%%%%%%%%%%%%%%%%%%%%%%%%%%%%%%%%%%
\heading{?`Qu{\'e} es la Inteligencia Artificial?}

$\ldots$ seg{\'u}n Kowalski:

La visi{\'o}n convencional: la soluci{\'o}n computacional de
tareas normalmente asociadas con la inteligencia humana. Por
ejemplo, jugar ajedrez.

Una visi{\'o}n m{\'a}s precisa: El uso de m{\'e}todos
sistem{\'a}ticos, al estilo de los humanos, aplicados a cualquier
tarea realizada por humanos o por m{\'a}quinas.

%%%%%%%%%%%% Slide %%%%%%%%%%%%%%%%%%%%%%%%%%%%%%%%%%%%%%%%%%%%%%%%%%%
\heading{?`Qu{\'e} NO es la Inteligencia Artificial?}

$\ldots$ seg{\'u}n Kowalski, son falacias comunes:

\blob que el objetivo de la inteligencia artificial es crear
m{\'a}quinas que ser{\'a}n m{\'a}s inteligentes que los humanos y
los tratar{\'a}n como sus mascotas.

\blob que la IA es factible porque, despu{\'e}s de todo, las
personas son s{\'o}lo m{\'a}quinas.


%%%%%%%%%%%% Slide %%%%%%%%%%%%%%%%%%%%%%%%%%%%%%%%%%%%%%%%%%%%%%%%%%%
\heading{?`Por qu{\'e} es importante la IA?}

$\ldots$ seg{\'u}n Kowalski:

\blob Las realizaciones y productos computacionales son m{\'a}s
f{\'a}ciles de desarrollar, mantener y manipular.

\blob Podr{\'\i}a proveernos de teor{\'\i}as de la inteligencia
humana simulables en el computador (o computadores).

\blob Teor{\'\i}as la inteligencia humana que podr{\'\i}an ser
aplicadas sistem{\'a}ticamente (es decir, como en ingenier{\'\i}a)
por las m{\'a}quinas o por los mismos humanos.

La mayor{\'\i}a de los m{\'e}todos de la inteligencia artificial
pueden ser vistos como versiones pr{\'a}cticas de la l{\'o}gica
formal.

%%%%%%%%%%%% Slide %%%%%%%%%%%%%%%%%%%%%%%%%%%%%%%%%%%%%%%%%%%%%%%%%%%
\heading{Actuando como humanos: el test de Turing}

Turing (1950) ``Computing machinery and intelligence":\\

\blob ``?`Pueden pensar las m{\'a}quinas?''\\
 $\longrightarrow$ ``?`Pueden las m{\'a}quinas comportarse inteligentemente?''\\

\blob Test operativo de la conducta inteligente: El juego de
imitaci{\'o}n del humano.

\vspace*{0.2in}

\epsfxsize=0.5\textwidth
\fig{\file{figures}{turing.ps}}

\blob predijo que para el 2000, una m{\'a}quina podr{\'\i}a, con
una probabilidad de 30\%, enga{\~n}ar a una persona no advertida
por unos 5 minutos.

%%%%%%%%%%%% Slide %%%%%%%%%%%%%%%%%%%%%%%%%%%%%%%%%%%%%%%%%%%%%%%%%%%
\heading{Actuando como humanos: el test de Turing}

No lo han logrado oficialmente. Pero prueben\\
\verb|https://www.kuki.ai/| y\\

\blob (S{\'o}lo por completitud, revisen este\\
\verb|https://sourceforge.net/projects/resumidor|\\
\verb|https://github.com/jacintodavila/resumidor|\\

\blob previ{\'o} todos los argumentos contra la IA en los siguientes 50 a{\~n}os\\

\blob sugiri{\'o} que los componentes principales de la IA son: el
conocimiento, el razonamiento, la comprensi{\'o}n ling{\"u}istica
y el aprendizaje\\

%%%%%%%%%%%% Slide %%%%%%%%%%%%%%%%%%%%%%%%%%%%%%%%%%%%%%%%%%%%%%%%%%%
\heading{Actuando como humanos: el test de Turing}

Un gran problema: El test de Turing no es \txr{reproducible},\\
ni \txr{constructivo}, \\
ni adecuado para \txr{an{\'a}lisis matem{\'a}tico}.\\

Y qu{\'e}?

%%%%%%%%%%%% Slide %%%%%%%%%%%%%%%%%%%%%%%%%%%%%%%%%%%%%%%%%%%%%%%%%%%
\heading{Pensando como los humanos: Cognitive Science}

1960s ``\txg{La revoluci{\'o}n cognitiva}'': la psicolog{\'\i}a
del procesamiento de la informaci{\'o}n reemplaz{\'o} la
perspectiva ortodoxa del \txb{conductivismo}.

Requiere teor{\'\i}as acerca de las actividades internas del
cerebro\al
 -- ?`Cual nivel de abstracci{\'o}n? ``\txg{Conocimiento}'' or ``\txg{?`circuiter{\'\i}a}''?\al
 -- ?`C{\'o}mo se validan? Algunos dicen que \nl
    1)Anticipando y evaluando la conducta de sujetos humanos (top-down)
    o 2) Identificaci{\'o}n directa de data neurol{\'o}gica (bottom-up)


%%%%%%%%%%%% Slide %%%%%%%%%%%%%%%%%%%%%%%%%%%%%%%%%%%%%%%%%%%%%%%%%%%
\heading{Pensando como los humanos: Cognitive Science}

Ambas m{\'e}todologias ( \txm{Cognitive Science} y \txb{Cognitive
Neuroscience}) se han distanciado ahora de la IA.

Ambas comparten con IA la siguiente caracter{\'\i}stica:{\em las
teor{\'\i}as actuales no explican (o producen) nada que se parezca
a la inteligencia humana}.

En consecuencia, los tres ``campos" comparten la misma
direcci{\'o}n. (frustaci{\'o}n?)

%%%%%%%%%%%% Slide %%%%%%%%%%%%%%%%%%%%%%%%%%%%%%%%%%%%%%%%%%%%%%%%%%%
\heading{Pensando racionalmente}

Las leyes del pensamiento.

\txr{Normativo} (o \txr{impositivo}) en lugar de \txg{descriptivo}
(pero correcto desde todo punto de vista).

Arist{\'o}teles: ?`cuando es correcto un argumento o proceso de
pensamiento?

En la antigua Grecia se desarrollaron varias formas de
\txr{l{\'o}gica}: \txr{notaciones} y \txr{reglas de inferencia};
que quiz{\'a}s pudieron haber contemplado la mecanizaci{\'o}n.

%%%%%%%%%%%% Slide %%%%%%%%%%%%%%%%%%%%%%%%%%%%%%%%%%%%%%%%%%%%%%%%%%%
\heading{Pensando racionalmente}

En cualquier caso, nuestra matem{\'a}tica y filosof'ia moderna le
heredan esa tradici{\'o}n a la IA.

Problemas:

1) Se dice que no toda conducta inteligente es precedida por un
razonamiento l{\'o}gico (?`es esto una objecci{\'o}n?).

2) Otros dicen: ?`Qu{\'e} prop{\'o}sito tiene pensar?. ?`Qu{\'e}
pensamientos \txr{debo} tener? (de nuevo, ?`es esto una
objecci{\'o}n?)

%%%%%%%%%%%% Slide %%%%%%%%%%%%%%%%%%%%%%%%%%%%%%%%%%%%%%%%%%%%%%%%%%%
\heading{Actuando racionalmente}

Conducta \txr{racional}: hacer lo correcto

Lo correcto: Aquello que se espera maximice el logro de las metas,
dada la informaci{\'o}n disponible. (?`Maximice?).

No implica necesariamente que se est{\'a} pensando en hacer lo
correcto --e.g., muchos reflejos como el parpadear son ``hacer lo
correct{\'o}'--- pero, se dice, que el pensamiento tiene que estar
al servicio de la acci{\'o}n racional.

Arist{\'o}teles (La Etica de Nic{\'o}medes):\al
  \txb{Todo arte y todas pregunta, y similarmente toda  acci{\'o}n
y procura, se supone que se orienta a alg{\'u}n beneficio}

%%%%%%%%%%%% Slide %%%%%%%%%%%%%%%%%%%%%%%%%%%%%%%%%%%%%%%%%%%%%%%%%%%
\heading{Raz{\'o}n}

\blob Raz{\'o}n: 1) Facultad de pensar, discurrir y juzgar. 2)
Facultad intelectual que permite actuar acert{\'a}damente o
distinguir lo bueno y verdadero de lo malo y falso. 3) Motivo.
[Diccionario Larousse].

\blob Reason: 1) a motive, cause or justification. 2) a fact
adduced or serving at this. 3) The intellectual faculty by which
conclusions are drawn from premises. 4) Sanity. 5) A faculty
transcending the understanding and providing a priori principles;
intuition. 7) Sense; what is right or practical or practicable:
``it stands to reason" is equivalent to ``it is evident or
logical" (The concise Oxford dictionary).

%%%%%%%%%%%% Slide %%%%%%%%%%%%%%%%%%%%%%%%%%%%%%%%%%%%%%%%%%%%%%%%%%%
\heading{Agentes racionales}

Un \txr{agente} es una entidad que percibe y act{\'u}a.

En este curso se aprende a dise{\~n}ar \txr{agentes racionales}
(a{\'u}n cuando parte del dise{\~n}o se fije en su ambiente)

Usando la matem{\'a}tica como lenguaje de especificaci{\'o}n,
podemos decir que un agente es una funci{\'o}n con dominio en una
``historia perceptual'' y rango en las acciones:
\[f: {\cal P}^* \rightarrow {\cal A}\]
Buscaremos, para cada tipo dado de ambiente y de tarea\\
el agente (o clase de agentes) que se comporta de la mejor manera.

Observaci{\'o}n: \txr{{\em Las limitaciones computaciones hacen
que la racionalidad perfecta sea inalcanzable}} $\rightarrow$
Dise{\~n}aremos el mejor \txr{programa} con los recursos dados.

O trataremos de ajustar algunos recursos para aumentar la
racionalidad.


%%%%%%%%%%%% Slide %%%%%%%%%%%%%%%%%%%%%%%%%%%%%%%%%%%%%%%%%%%%%%%%%%%
\heading{La prehistoria de la IA}

\begin{tabular}{ll}
\txb{Filosof{\'\i}a}    & La l{\'o}gica como un m{\'e}todo para razonar\\
\tabbot                 & La mente como un sistema f{\'\i}sico\\
\tabbot                 & Los fundamentos del aprendizaje,\\ 
\tabbot                 & del lenguaje y de la racionalidad\\
\txb{Matem{\'a}tica}    & Representaciones formales y demostraciones\\
\tabbot                 & Algoritmos, computaci{\'o}n, \\
\tabbot                 & (in)decidibilidad, (in)tratabilidad,\\
\tabbot                 & probabilidad\\
\txb{Psicolog{\'\i}a}   & Adaptaci{\'o}n\\
\tabbot                 & Los fen{\'o}menos de la percepci{\'o}n\\
\tabbot                 & y el control motor\\
\tabbot                 & psicof{\'\i}sica, etc.\\
\txb{Econom{\'\i}a}     & Teor{\'\i}a formal de \\
\tabbot                 & decisiones racionales\\
\txb{Ling{\"u}istica}       & Representaci{\'o}n del conocimiento,\\
\tabbot                 & gram{\'a}ticas\\
\txb{Neurociencia}      & el substrato f{\'\i}sico \\
\tabbot                 & vers{\'a}til para la actividad mental\\
\txb{Teor{\'\i}a de Control}& Sistemas homeost{\'a}ticos, estabilidad\\
\tabbot                 & Agentes simples y {\'o}ptimos
\end{tabular}

%%%%%%%%%%%% Slide %%%%%%%%%%%%%%%%%%%%%%%%%%%%%%%%%%%%%%%%%%%%%%%%%%%
\heading{La historia ``enlatada" de la IA}
\begin{tabular}{ll}
\txg{1943}       & McCulloch \& Pitts: Modelo del cerebro con circuitos booleanos\\
\txg{1950}       & Turing: ``Computing Machinery and Intelligence"\\
\txg{1952--69}   & Auto frenes{\'\i}! \\
\txg{1950s}      & Primeros programas de IA: Newell \& Simon's Logic Theorist\\
\txg{1956}       & Conferencia de Dartmouth: Surge el nombre ``Artificial Intelligence"\\
\txg{1965-72}    & El m{\'e}todo de resoluci{\'o}n de Robinson y luego Kowalski\\
                 & (un algoritmo completo para razonamiento l{\'o}gico) -- PROLOG\\
\txg{1966--74}   & La IA se encuentra con la complejidad computacional\\
           & La investigaci{\'o}n en redes neuronales casi desaparece\\
\txg{1969--79}   & Los primeros sistemas basados en conocimiento\\
\txg{1980--88}   & Se popularizan los sistemas expertos\\
\txg{1988--93}   & Se desinflan los sistemas expertos: ``El primer invierno de la IA"\\
\txg{1985--95}   & Vuelven las redes neuronales\\
\txg{1988}--     & Reaparece la probabilidad: aumenta el rigor t{\'e}cnico\\
           & ``Nouvelle AI": Vida Artificial, AG, soft computing\\
\txg{1995}--     & Agentes!!!!\\ 
\txg{2000}--     & Machine Learning (Deep learning) $\ldots$
\end{tabular}

%%%%%%%%%%%% Slide %%%%%%%%%%%%%%%%%%%%%%%%%%%%%%%%%%%%%%%%%%%%%%%%%%%
\heading{La historia gruesa de la IA}
\begin{tabular}{ll}
\txg{1950s y 1960s}                     & B{\'u}squeda, juegos y redes neuronales\\
\txg{1960 y principios de '70s}         & Prueba autom{\'a}tica de teoremas\\
\txg{Los '70s y principios de '80s}     & Representaciones del conocimiento\\
                                        & Programaci{\'o}n l{\'o}gica\\
                                        & Sistemas expertos\\
\txg{Los '80s}                          & El proyecto de 5ta Generaci{\'o}n Japon{\'e}s\\
                                        & L{\'o}gicas para razonamiento por omisi{\'o}n\\
                                        & Razonamiento basado en casos\\
\txg{Finales 80s al presente}           & Redes neuronales\\
                                        & Aprendizaje (miner{\'\i}a de datos)\\
                                        & Sistemas reactivos (vs. racionales)\\
                                        & Inteligencia Artificial Distribuida\\
                                        & Sistemas multiagentes\\
                                        & Sociedades artificiales\\
\txg{Siglo XXI}                         & Redes neuronales\\
                                        & Aprendizaje (miner{\'\i}a de muchos datos)\\
\end{tabular}

%%%%%%%%%%%% Slide %%%%%%%%%%%%%%%%%%%%%%%%%%%%%%%%%%%%%%%%%%%%%%%%%%%
\heading{El estado del arte}

?`Cu{\'a}les de estas se pueden hacer en estos momentos?

\blob Jugar ping-pong decentemente \\
\blob Conducir en la monta{\~n}a \\
\blob Conducir en el centro de una ciudad \\
\blob Comprar el mercado de la semana \\
\blob Comprar el mercado de la semana en la web \\
\blob Jugar un ``bridg{\'e}' decentemente \\
\blob Descubrir y probar nuevos teoremas matem'aticos \\
\blob Escribir una historia divertida con toda la intenci{\'o}n \\
\blob Proveer consejo legal competente en una {\'a}rea especializada de la ley\\
\blob Traducir en tiempo real\\
\blob Realizar una intervenci{\'o}n quir{\'u}jica complicada
\blob Interactuar con fluidez con humanos

%%%%%%%%%%%% Slide %%%%%%%%%%%%%%%%%%%%%%%%%%%%%%%%%%%%%%%%%%%%%%%%%%%
\heading{El estado de nuestro arte}

\blob Sistemas multiagentes (Bioinformantes)\\
\verb|https://sourceforge.net/projects/simulants|\\
\blob Simulaci{\'o}n de sistemas multiagentes (GALATEA)\\
\verb|http://galatea.sourceforge.net/|\\
\verb|http://gloria.sourceforge.net/|\\
\blob Ling{\"u}istica computacional (Elizaul, Resumidor, Metaforizador)\\
\verb|https://sourceforge.net/projects/resumidor|\\
\verb|https://github.com/jacintodavila/resumidor|\\
\verb|https://www.logicalcontracts.com/|\\

\blob ?`Qu{\'e} sabemos hacer localmente?

%%%%%%%%%%%% Slide %%%%%%%%%%%%%%%%%%%%%%%%%%%%%%%%%%%%%%%%%%%%%%%%%%%
\heading{Historias divertidas sin intenci{\'o}n}

pero en ingl{\'e}s

One day Joe Bear was hungry. He asked his friend Irving Bird where some
honey was. Irving told him there was a beehive in the oak tree. Joe threatened
to hit Irving if he didn't tell him where some honey was.

Henry Squirrel was thirsty. He walked over to the river bank where his
good friend Bill Bird was sitting. Henry slipped and fell in the
river. Gravity drowned.

Once upon a time there was a dishonest fox and a vain crow. One day the
crow was sitting in his tree, holding a piece of cheese in his mouth. He
noticed that he was holding the piece of cheese. He became hungry, and
swallowed the cheese. The fox walked over to the crow. The end.

%%%%%%%%%%%% Slide %%%%%%%%%%%%%%%%%%%%%%%%%%%%%%%%%%%%%%%%%%%%%%%%%%%
\heading{Historias divertidas sin intenci{\'o}n}

pero en ingl{\'e}s

Joe Bear was hungry. He asked Irving Bird where some honey was.
Irving refused to tell him, so Joe offered to bring him a worm if
he'd tell him where some honey was.  Irving agreed. But Joe didn't
know where any worms were, so he asked Irving, who refused to say.
So Joe offered to bring him a worm if he'd tell him where a worm
was.  Irving agreed. But Joe didn't know where any worms were, so
he asked Irving, who refused to say. So Joe offered to bring him a
worm if he'd tell him where a worm was $\ldots$

%%%%%%%%%%%% Slide %%%%%%%%%%%%%%%%%%%%%%%%%%%%%%%%%%%%%%%%%%%%%%%%%%%
\heading{Historias divertidas con intenci{\'o}n}

 parcial (pero en espa{\~n}ol):

\begin{verbatim}
?- modelo.
   actantes:
      sujetos:
         suj_cogn:
            destinadores:
               desd_suj:
                  --> la_virgen_de_la_chinita
               anti_desd:
                  --> gente_del_barrio

            destinatarios:
               dest_suj:
                  --> mari-mach�
               anti_dest:
                  --> macky_narr
\end{verbatim}

%%%%%%%%%%%% Slide %%%%%%%%%%%%%%%%%%%%%%%%%%%%%%%%%%%%%%%%%%%%%%%%%%%
\heading{Historias divertidas con intenci{\'o}n}

 parcial (pero en espa{\~n}ol).

\begin{verbatim}

         suj_prag:
            sujeto:
               heroe:
                  --> ernesto_ernesto
               adyuvante:
                  --> el_negro_mesopotamio

            anti_suj:
               traidor:
                  --> el_jefe_de_la_mafia
               oponente:
                  --> la_mafia
\end{verbatim}

%%%%%%%%%%%% Slide %%%%%%%%%%%%%%%%%%%%%%%%%%%%%%%%%%%%%%%%%%%%%%%%%%%
\heading{Historias divertidas con intenci{\'o}n}

 parcial (pero en espa{\~n}ol).


\begin{verbatim}

      objetos:
         obj_mod:
            obj_mod_cogn:
               --> voluntad_y_prohibicion
            obj_mod_prag:
               --> discrecion_e_incapacidad

         obj_prag:
            obj_valor:
               --> el_amor_sincero
            obj_uso:
               --> la_carta_secreta
\end{verbatim}


%%%%%%%%%%%% Slide %%%%%%%%%%%%%%%%%%%%%%%%%%%%%%%%%%%%%%%%%%%%%%%%%%%
\heading{Ensayos inteligentes}

\blob La inteligencia y los sentimientos \\
\blob La inteligencia y lo subjetivo \\
\blob La inteligencia individual y la inteligencia colectiva\\
\blob La inteligencia y racionalidad acotada\\
\blob La inteligencia y el mercado\\
\blob La inteligencia y los servicios p{\'u}blicos\\
\blob La inteligencia natural y la inteligencia artificial\\
\blob La inteligencia y el tiempo\\
\blob Los sistemas cercanos que son muy inteligentes\\
\blob Los sistemas cercanos que podr{\'\i}an ser m{\'a}s inteligentes\\
\blob La inteligencia y la democracia\\
\blob La inteligencia y la sabiduria\\
\blob La inteligencia y la evoluci{\'o}n\\
\blob La inteligencia y la felicidad\\
\blob La inteligencia y la justicia\\
\blob La inteligencia, la moralidad y la corrupci{\'o}n\\
\blob La inteligencia y la complejidad\\
\blob La inteligencia y el lenguaje\\


\end{huge}
%\end{landscape}
\end{document}
