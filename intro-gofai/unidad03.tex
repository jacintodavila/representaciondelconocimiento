\documentclass{article}
\usepackage{fleqn}
\usepackage{epsf}
\usepackage[dvips]{color}
\usepackage{aima2e-slides}

\begin{document}

\begin{huge}
\titleslide{B{\'u}squeda y resoluci{\'o}n de problemas}{Cap{\'\i}tulo 3, Secciones 1--5}

\sf

%%%%%%%%%%%% Slide %%%%%%%%%%%%%%%%%%%%%%%%%%%%%%%%%%%%%%%%%%%%%%%%%%%
% \heading{Reminders}

%%%%%%%%%%%% Slide %%%%%%%%%%%%%%%%%%%%%%%%%%%%%%%%%%%%%%%%%%%%%%%%%%%
\heading{Contenido}

\blob Agentes solucionadores de problemas

\blob Tipos de problemas

\blob Formulaci{\'o}n de problemas

\blob Problemas de ejemplo

\blob Algoritmos b{\'a}sicos de b{\'u}squeda


%%%%%%%%%%%% Slide %%%%%%%%%%%%%%%%%%%%%%%%%%%%%%%%%%%%%%%%%%%%%%%%%%%
\heading{Agentes solucionadores de problemas}

Forma restringida de una agente universal:

\input{\file{algorithms}{simple-PS-agent-algorithm}}

Note que esta es una forma de resoluci{\'o}n sin conexi{\'o}n con
el mundo (offline).

%%%%%%%%%%%% Slide %%%%%%%%%%%%%%%%%%%%%%%%%%%%%%%%%%%%%%%%%%%%%%%%%%%
\heading{Agentes solucionadores de problemas}

La soluci{\'o}n es ejecutada ``con los ojos cerrados". El agente
asume que ese es todo el conocimiento que se tiene (y se puede
tener) del ambiente (conocimiento completo). Una suposici{\'o}n
normalmente equivocada, pero suficiente en muchos casos.


%%%%%%%%%%%% Slide %%%%%%%%%%%%%%%%%%%%%%%%%%%%%%%%%%%%%%%%%%%%%%%%%%%
\heading{El ejemplo de Ruman{\'\i}a}


Estoy de vacaciones en Ruman{\'\i}a. En estos momentos estoy en
Arad.\\
Mi avi{\'o}n de vuelta a Venezuela sale ma{\~n}ana de Bucarest.

\txb{Formule la meta}:\nl Estar en Bucarest.

\txb{Formule el problema}:\nl \txr{\em Estados}: Las diferentes
ciudades \nl \txr{\em Acciones}: conducir mi auto entre las
ciudades.

\txb{Encuentra la soluci{\'o}n}:\nl Una secuencia de ciudades,
e.g. Arad, Sibiu, Fagaras, Bucarest.


%%%%%%%%%%%% Slide %%%%%%%%%%%%%%%%%%%%%%%%%%%%%%%%%%%%%%%%%%%%%%%%%%%
\heading{El ejemplo de Ruman{\'\i}a}


\epsfxsize=\maxfigwidth
\fig{\file{figures}{romania-distances.ps}}


%%%%%%%%%%%% Slide %%%%%%%%%%%%%%%%%%%%%%%%%%%%%%%%%%%%%%%%%%%%%%%%%%%
\heading{Tipos de problemas}

\txb{Deterministico, completamente observable} $\Longrightarrow$
\txr{\em problema de un s{\'o}lo estado}\nl
    El agente sabe exactamente a cual estado arribar{\'a}. La
    soluci{\'o}n es una secuencia de estados (o acciones).

\txb{No observable} $\Longrightarrow$ \txr{\em problema de
m{\'u}ltiples estados}\nl
    El agente puede no tener idea de donde est{\'a}. La soluci{\'o}n, si
    existe, es una colecci{\'o}n de estados posibles a donde
    arribar{\'\i}a. El agente si conoce el efecto de sus acciones.
    Piensen cuan exigente es esta {\'u}ltima condici{\'o}n.

\txb{No-deterministico o parcialmente observable}
$\Longrightarrow$ \txr{\em problem de contigencias}\nl Los
perceptos proveen \txr{\em nueva} informaci{\'o}n sobre el estado
actual\nl La soluci{\'o}n es un \txr{\em tree} or una \txr{\em
pol{\'\i}tica} (varias ramas del {\'a}rbol). Normalmente se
solapan b{\'u}squeda y ejecuci{\'o}n.

\txb{Espacio de estados desconocido} $\Longrightarrow$ \txr{\em
problem de exploraci{\'o}n} (``online'')


%%%%%%%%%%%% Slide %%%%%%%%%%%%%%%%%%%%%%%%%%%%%%%%%%%%%%%%%%%%%%%%%%%
\heading{El ejemplo del mundo de la aspiradora}

\begin{tabular}{lr}
\hbox{\begin{minipage}[t]{0.6\textwidth} \txb{Estado simple},
comienza en \#5. \q{Soluci{\'o}n}
\end{minipage}}
&
\hbox{\begin{minipage}[t]{0.4\maxfigwidth}
\epsfxsize=0.4\maxfigwidth
\raisebox{-0.35\maxfigwidth}[0pt][0pt]{\epsffile{\file{figures}{vacuum2-space.ps}}}
\end{minipage}}
\end{tabular}


%%%%%%%%%%%% Slide %%%%%%%%%%%%%%%%%%%%%%%%%%%%%%%%%%%%%%%%%%%%%%%%%%%
\heading{El ejemplo del mundo de la aspiradora}

\begin{tabular}{lr}
\hbox{\begin{minipage}[t]{0.6\textwidth} \txb{Estado simple},
comienza en \#5. \q{Soluci{\'o}n}
\\
$[Right,Suck]$
\\[0.5\baselineskip]
\txb{Estado m{\'u}ltiple}, comienza en $\{1,2,3,4,5,6,7,8\}$\\
e.g., $Right$ llega a $\{2,4,6,8\}$. \q{Soluci{\'o}n}
\end{minipage}}
&
\hbox{\begin{minipage}[t]{0.4\maxfigwidth}
\epsfxsize=0.4\maxfigwidth
\raisebox{-0.35\maxfigwidth}[0pt][0pt]{\epsffile{\file{figures}{vacuum2-space.ps}}}
\end{minipage}}
\end{tabular}


%%%%%%%%%%%% Slide %%%%%%%%%%%%%%%%%%%%%%%%%%%%%%%%%%%%%%%%%%%%%%%%%%%
\heading{El ejemplo del mundo de la aspiradora}

\begin{tabular}{lr}
\hbox{\begin{minipage}[t]{0.6\textwidth} \txb{Estado simple},
comienza en \#5. \q{Soluci{\'o}n}
\\
$[Right,Suck]$
\\[0.5\baselineskip]
\txb{Estado m{\'u}ltiple}, comienza en $\{1,2,3,4,5,6,7,8\}$\\
e.g., $Right$ llega a $\{2,4,6,8\}$. \q{Soluci{\'o}n}
\\
$[Right,Suck,Left,Suck]$
\\[0.5\baselineskip]
\txb{Contingencia}, comienza en \#5\\
La ley de Murphy: $Suck$ puede ensuciar una alfombra limpia\\
Percepci{\'o}n local : sucio y ubicaci{\'o}n solamente.\\
\q{Soluci{\'o}n}
\end{minipage}}
&
\hbox{\begin{minipage}[t]{0.4\maxfigwidth}
\epsfxsize=0.4\maxfigwidth
\raisebox{-0.35\maxfigwidth}[0pt][0pt]{\epsffile{\file{figures}{vacuum2-space.ps}}}
\end{minipage}}
\end{tabular}


%%%%%%%%%%%% Slide %%%%%%%%%%%%%%%%%%%%%%%%%%%%%%%%%%%%%%%%%%%%%%%%%%%
\heading{El ejemplo del mundo de la aspiradora}

\begin{tabular}{lr}
\hbox{\begin{minipage}[t]{0.6\textwidth} \txb{Estado simple},
comienza en \#5. \q{Soluci{\'o}n}
\\
$[Right,Suck]$
\\[0.5\baselineskip]
\txb{Estado m{\'u}ltiple}, comienza en $\{1,2,3,4,5,6,7,8\}$\\
e.g., $Right$ llega a $\{2,4,6,8\}$. \q{Soluci{\'o}n}
\\
$[Right,Suck,Left,Suck]$
\\[0.5\baselineskip]
\txb{Contingencia}, comienza en \#5\\
La ley de Murphy: $Suck$ puede ensuciar una alfombra limpia\\
Percepci{\'o}n local : sucio y ubicaci{\'o}n solamente.\\
\q{Soluci{\'o}n}
\\
$[Right,\k{si}\ dirt\ \k{entonces}\ Suck]$
\end{minipage}}
&
\hbox{\begin{minipage}[t]{0.4\maxfigwidth}
\epsfxsize=0.4\maxfigwidth
\raisebox{-0.35\maxfigwidth}[0pt][0pt]{\epsffile{\file{figures}{vacuum2-space.ps}}}
\end{minipage}}
\end{tabular}


%%%%%%%%%%%% Slide %%%%%%%%%%%%%%%%%%%%%%%%%%%%%%%%%%%%%%%%%%%%%%%%%%%
\heading{Formulaci{\'o}n del problema de un solo estado}

Un \txr{\em problema} se define con los siguiente 4 items:

\txb{\txr{\em un estado inicial}} \quad e.g., ``en Arad''

\txb{\txr{\em una funci{\'o}n succesor}} $S(x)$ = un conjunto de
pares acci{\'o}n-estado asociados a un destino \nl e.g., $S(Arad)
= \{\<Arad\rightarrow Zerind, Zerind\>, \ldots \}$

\txb{\txr{\em un test de meta}}, que puede ser \txr{\em
expl{\'\i}cito}, e.g., $x$ = ``at Bucharest''\nl o \txr{\em
impl{\'\i}cito}, e.g., $NoDirt(x)$

\txb{\txr{\em funci{\'o}n de costos}} (aditiva)\nl e.g., la suma
de distancias, el n{\'u}mero  de acciones ejecutadas, etc.\nl
$c(x,a,y)$ es el \txb{costo de paso}, que se asume siempre $\geq
0$

Una \txr{\em soluci{\'o}n} es una secuencia de acciones\\
que llevan al agente y a su ambiente desde el estado inicial hasta
estado final

%%%%%%%%%%%% Slide %%%%%%%%%%%%%%%%%%%%%%%%%%%%%%%%%%%%%%%%%%%%%%%%%%%
\heading{Seleccionando un espacio de estados}

El mundo se presenta de muchas maneras y con muchos detalles.

Enfrentamos una enorme complejidad \nl $\Rightarrow$ el espacio de
estados debe ser una \txr{\em abstracci{\'o}n} de esa complejidad
para poder emprender la resoluci{\'o}n de los problemas.

Un estado (abstracto) en nuestro modelo = conjunto de estados
reales.

Acci{\'o}n (abstracta)  = agregado o combinaci{\'o}n de acciones
reales\nl
   e.g., ``Arad $\rightarrow$ Zerind'' representa un conjunto
   complejo de rutas possibles, desvios, paradas de descanso,
   etc.\\

Para garantizar que tal acci{\'o}n se puede realizar, es preciso
que \txb{cualquier} estado real ``en Arad'' pueda conducir a
\txr{\em alg{\'u}n} estado real en ``en Zerind''

Soluciones (abstractas) =\nl
    conjunto de caminos reales que son soluciones en el mundo
    real.

Cada una de las acciones abstractas debe ser ``m{\'a}s f{\'a}cil''
que el problema original.

%%%%%%%%%%%% Slide %%%%%%%%%%%%%%%%%%%%%%%%%%%%%%%%%%%%%%%%%%%%%%%%%%%
\heading{\small Ejemplo: el grafo del espacio de estados del mundo
de la aspiradora}

\epsfxsize=0.8\textwidth
\fig{\file{figures}{vacuum2-paths.ps}}

\q{estados}\\
\q{acciones}\\
\q{test de meta}\\
\q{costos}


%%%%%%%%%%%% Slide %%%%%%%%%%%%%%%%%%%%%%%%%%%%%%%%%%%%%%%%%%%%%%%%%%%
\heading{\small Ejemplo: el grafo del espacio de estados del mundo
de la aspiradora}

\epsfxsize=0.8\textwidth
\fig{\file{figures}{vacuum2-paths.ps}}

\q{estados}: n{\'u}mero entero para indicar sucio y ubicaci{\'o}n del robot (pero ignora  \txr{\em cantidad} de sucio)\\
\q{acciones}: $Left$, $Right$, $Suck$, $NoOp$\\
\q{test de meta}: no hay sucio en ninguna ubicaci{\'o}n\\
\q{costo}: cuenta 1 por cada acci{\'o}n (0 por $NoOp$)



%%%%%%%%%%%% Slide %%%%%%%%%%%%%%%%%%%%%%%%%%%%%%%%%%%%%%%%%%%%%%%%%%%
\heading{Example: The 8-puzzle}

\epsfxsize=0.6\textwidth
\fig{\file{figures}{8puzzle.ps}}

\q{estados}\\
\q{acciones}\\
\q{test de meta}\\
\q{costos}




%%%%%%%%%%%% Slide %%%%%%%%%%%%%%%%%%%%%%%%%%%%%%%%%%%%%%%%%%%%%%%%%%%
\heading{Ejemplo: fichero de 8}

\epsfxsize=0.6\textwidth
\fig{\file{figures}{8puzzle.ps}}

\q{estados}: ubicaci{\'o}n de las fichas en enteros (ignora posiciones intermedias)\\
\q{acciones}: mover el vaci{\'o} izquierda, derecha, arriba, abajo (ignore destranque, etc.)\\
\q{test de meta}: = cualquier configuraci{\'o}n predefinida\\
\q{costos}: 1 por movimiento.

[Nota: La soluci{\'o}n {\'o}ptima de problemas como el fichero de
n es NP-Compleja]




%%%%%%%%%%%% Slide %%%%%%%%%%%%%%%%%%%%%%%%%%%%%%%%%%%%%%%%%%%%%%%%%%%
\heading{Ejemplo: El robot de ensamblaje}

\epsfxsize=0.5\textwidth
\fig{\file{figures}{stanford-arm+blocks.ps}}

\q{estados}: las coordenadas (valores reales) de cada coyuntura
del robot.\nl Las partes del objeto que ser{\'a}n ensambladas.

\q{acciones}: movimientos continuos de las coyunturas del robot
(m{\'a}s algunas acciones discretas para cambiar sus
trayectorias).

\q{test de meta}: ensamblaje completo.

\q{costos}: tiempo de ejecuci{\'o}n (energia empleada??)


%%%%%%%%%%%% Slide %%%%%%%%%%%%%%%%%%%%%%%%%%%%%%%%%%%%%%%%%%%%%%%%%%%
\heading{Algoritmos de b{\'u}squeda en un {\'a}rbol}

Idea b{\'a}sica:\al exploraci{\'o}n simulada ``offline'', del
espacio de estados\al para lo cual se generan sucesores de los
estados ya explorados\nnl, una t{\'e}cnica tamb{\'\i}en conocida
como \txr{\em expandir} los estados.


\input{\file{algorithms}{informal-ts-algorithm}}




%%%%%%%%%%%% Slide %%%%%%%%%%%%%%%%%%%%%%%%%%%%%%%%%%%%%%%%%%%%%%%%%%%
\heading{Ejemplo de b{\'u}squeda en un {\'a}rbol}

\vspace*{0.3in}

\epsfxsize=1.0\textwidth
\fig{\sfile{figures}{search-map1.ps}}

%%%%%%%%%%%% Overlay %%%%%%%%%%%%%%%%%%%%%%%%%%%%%%%%%%%%%%%%%%%%%%%%%%%
\pheading{Ejemplo de b{\'u}squeda en un {\'a}rbol}

\vspace*{0.3in}

\epsfxsize=1.0\textwidth
\fig{\sfile{figures}{search-map2.ps}}


%%%%%%%%%%%% Overlay %%%%%%%%%%%%%%%%%%%%%%%%%%%%%%%%%%%%%%%%%%%%%%%%%%%
\pheading{Ejemplo de b{\'u}squeda en un {\'a}rbol}

\vspace*{0.3in}

\epsfxsize=1.0\textwidth
\fig{\sfile{figures}{search-map3.ps}}


%%%%%%%%%%%% Slide %%%%%%%%%%%%%%%%%%%%%%%%%%%%%%%%%%%%%%%%%%%%%%%%%%%
\heading{Implementaci{\'o}n: estados vs.{\~n}nodos}

Un \txr{\em estado} es una (representaci{\'o}n de) una
configuraci{\'o}n
f{\'\i}sica\\

Un \txr{\em node} es una estructura de data parte constitutiva de
un {\'a}rbol de b{\'u}squeda\nl que incluye \txr{\em padres},
\txr{\em
hijos}, \txr{\em profundidad} y \txr{\em costo del camino} $g(x)$\\
Los \txr{\em estados} no tienen padres, hijos, profundidad o costo
del camino!.

\epsfxsize=0.65\textwidth \fig{\file{figures}{state-vs-node.ps}}
La funci{\'o}n {\sc Expand} crea nuevos nodos, completando los
diversos campos de la estructura y usando {\sc SuccesorFn} del
problema para crear los estados correspondientes.


%%%%%%%%%%%% Slide %%%%%%%%%%%%%%%%%%%%%%%%%%%%%%%%%%%%%%%%%%%%%%%%%%%
\heading{Implementaci{\'o}n: b{\'u}squeda general en un {\'a}rbol}

\input{\file{algorithms}{tree-search-algorithm}}


%%%%%%%%%%%% Slide %%%%%%%%%%%%%%%%%%%%%%%%%%%%%%%%%%%%%%%%%%%%%%%%%%%
\heading{Estrategias de b{\'u}squeda}

Cada estrategia se define en la forma de seleccionar \txr{\em el
orden el que se ``expanden'' los (caminos de) nodos}


Las estrategias se evaluan de acuerdo a los siguientes
aspectos:\nl

\txb{completitud}--- ?`Acaso consigue siempre una soluci{\'o}n, si
nuna existe?\nl

\txb{complejidad en tiempo}--- N{\'u}mero de nodos
generados/expandidos\nl

\txb{complejidad en memoria}--- M{\'a}ximo n{\'u}mero de nodos en
memoria\nl

\txb{optimalidad}---?`Consigue siempre una soluci{\'o}n de
m{\'\i}nimo costo?

%%%%%%%%%%%% Slide %%%%%%%%%%%%%%%%%%%%%%%%%%%%%%%%%%%%%%%%%%%%%%%%%%%
\heading{Estrategias de b{\'u}squeda}

Las complejidades de tiempo y espacio se miden en t{\'e}rminos de:

$b$---factor de ``enramaj{\'e}' (ramificaci{\'o}n) m{\'a}ximo del
{\'a}rbol de b{\'u}squeda\nl

$d$---profundidad de la soluci{\'o}n de menor costo\nl

$m$---profundidad m{\'a}xima del espacio de estados (que puede ser
$\infty$)


%%%%%%%%%%%% Slide %%%%%%%%%%%%%%%%%%%%%%%%%%%%%%%%%%%%%%%%%%%%%%%%%%%
\heading{Estrategias de b{\'u}squeda no informadas}

Las estrategias \txr{\em no informadas} s{\'o}lo usan la
informaci{\'o}n
disponible en la definici{\'o}n del problema\\

B{\'u}squeda primero en anchura/Breadth-first search

B{\'u}squeda de costo uniforme/Uniform-cost search

B{\'u}squeda primero en profundidad/Depth-first search

B{\'u}squeda de profundidad limitada/Depth-limited search

B{\'u}squeda de profundizamiento iterativo/Iterative deepening
search


%%%%%%%%%%%% Slide %%%%%%%%%%%%%%%%%%%%%%%%%%%%%%%%%%%%%%%%%%%%%%%%%%%
\heading{B{\'u}squeda primero en anchura}

Expanda el nodo no expandido menos profundo.

\txb{Implementaci{\'o}n}:\nl \v{fringe} es una cola FIFO: los
nuevos sucesores se agregan al final.

\epsfxsize=0.5\textwidth
\fig{\sfile{figures}{bfs-progress1.ps}}

%%%%%%%%%%%% Slide %%%%%%%%%%%%%%%%%%%%%%%%%%%%%%%%%%%%%%%%%%%%%%%%%%%
\heading{B{\'u}squeda primero en anchura}

Expanda el nodo no expandido menos profundo.

\txb{Implementaci{\'o}n}:\nl \v{fringe} es una cola FIFO: los
nuevos sucesores se agregan al final.

\epsfxsize=0.5\textwidth
\fig{\sfile{figures}{bfs-progress2.ps}}

%%%%%%%%%%%% Slide %%%%%%%%%%%%%%%%%%%%%%%%%%%%%%%%%%%%%%%%%%%%%%%%%%%
\heading{B{\'u}squeda primero en anchura}

Expanda el nodo no expandido menos profundo.

\txb{Implementaci{\'o}n}:\nl \v{fringe} es una cola FIFO: los
nuevos sucesores se agregan al final.

\epsfxsize=0.5\textwidth
\fig{\sfile{figures}{bfs-progress3.ps}}

%%%%%%%%%%%% Slide %%%%%%%%%%%%%%%%%%%%%%%%%%%%%%%%%%%%%%%%%%%%%%%%%%%
\heading{B{\'u}squeda primero en anchura}

Expanda el nodo no expandido menos profundo.

\txb{Implementaci{\'o}n}:\nl \v{fringe} es una cola FIFO: los
nuevos sucesores se agregan al final.

\epsfxsize=0.5\textwidth
\fig{\sfile{figures}{bfs-progress4.ps}}


%%%%%%%%%%%% Slide %%%%%%%%%%%%%%%%%%%%%%%%%%%%%%%%%%%%%%%%%%%%%%%%%%%
\heading{Propiedades de la b{\'u}squeda primero en anchura}

\q{Completa}

%%%%%%%%%%%% Slide %%%%%%%%%%%%%%%%%%%%%%%%%%%%%%%%%%%%%%%%%%%%%%%%%%%
\heading{Propiedades de la b{\'u}squeda primero en anchura}

\q{Completa} Si (si $b$ is finito)

\q{Tiempo}

%%%%%%%%%%%% Slide %%%%%%%%%%%%%%%%%%%%%%%%%%%%%%%%%%%%%%%%%%%%%%%%%%%
\heading{Propiedades de la b{\'u}squeda primero en anchura}

\q{Completa} Si (si $b$ is finito)

\q{Tiempo} $1+b+b^2+b^3+\ldots +b^d + b(b^d-1)= O(b^{d+1})$, es
decir, es exponencial.{\~n}en $d$

\q{Espacio}

%%%%%%%%%%%% Slide %%%%%%%%%%%%%%%%%%%%%%%%%%%%%%%%%%%%%%%%%%%%%%%%%%%
\heading{Propiedades de la b{\'u}squeda primero en anchura}

\q{Completa} Si (si $b$ is finito)

\q{Tiempo} $1+b+b^2+b^3+\ldots +b^d + b(b^d-1)= O(b^{d+1})$, es
decir, es exponencial.{\~n}en $d$

\q{Espacio} $O(b^{d+1})$ (Guarda todos los nodos en memoria)

\q{Optimalidad}

%%%%%%%%%%%% Slide %%%%%%%%%%%%%%%%%%%%%%%%%%%%%%%%%%%%%%%%%%%%%%%%%%%
\heading{Propiedades de la b{\'u}squeda primero en anchura}

\q{Completa} S{\'\i} (si $b$ is finito)

\q{Tiempo} $1+b+b^2+b^3+\ldots +b^d + b(b^d-1)= O(b^{d+1})$, es
decir, es exponencial.{\~n}en $d$

\q{Espacio} $O(b^{d+1})$ (Guarda todos los nodos en memoria)

\q{Optimalidad} S{\'\i} (is el cost = 1 por paso); en general no
es {\'o}ptima.

El \txr{\em espacio} es el gran problema aqu{\'\i}. La estrategia
puede generar nodos a una taza de 10MB/seg\nl Es decir, en 24
horas son 860 GB.


%%%%%%%%%%%% Slide %%%%%%%%%%%%%%%%%%%%%%%%%%%%%%%%%%%%%%%%%%%%%%%%%%%
\heading{B{\'u}squeda de costo uniforme}

Expande el nodo no expandido de menor costo.

\txb{Implementaci{\'o}n}:\nl \v{fringe} = cola ordenada por el
costo del camino.

Es equivalente a la b{\'u}squeda primero en anchura si todos los
pasos son del mismo costo.

\q{Completa} S{\'\i}, si el costo del paso es $\geq \epsilon$

\q{Tiempo} \# de nodos con $g \leq {}$ costo de la soluci{\'o}n
{\'o}ptima, $O(b^{\ceiling{C^*/\epsilon}})$\al donde $C^*$ es el
costo de la soluci{\'o}n {\'o}ptima.

\q{Espacio} \# de nodos con $g \leq {}$ costo de la soluci{\'o}n
{\'o}ptima $O(b^{\ceiling{C^*/\epsilon}})$

\q{Optimalidad} S{\'\i}---los nodos se expanden en el orden
creciente de $g(n)$


%%%%%%%%%%%% Slide %%%%%%%%%%%%%%%%%%%%%%%%%%%%%%%%%%%%%%%%%%%%%%%%%%%
\heading{B{\'u}squeda primero en profundidad}

Expande el nodo no expandido de major profundidad.

\txb{Implementaci{\'o}n}:\nl \v{fringe} = cola LIFO , es decir,
coloca los sucesores al frente

\epsfxsize=0.5\textwidth
\fig{\sfile{figures}{dfs-progress01.ps}}


%%%%%%%%%%%% Slide %%%%%%%%%%%%%%%%%%%%%%%%%%%%%%%%%%%%%%%%%%%%%%%%%%%
\heading{B{\'u}squeda primero en profundidad}

Expande el nodo no expandido de major profundidad.

\txb{Implementaci{\'o}n}:\nl \v{fringe} = cola LIFO , es decir,
coloca los sucesores al frente

\epsfxsize=0.5\textwidth
\fig{\sfile{figures}{dfs-progress02.ps}}


%%%%%%%%%%%% Slide %%%%%%%%%%%%%%%%%%%%%%%%%%%%%%%%%%%%%%%%%%%%%%%%%%%
\heading{B{\'u}squeda primero en profundidad}

Expande el nodo no expandido de major profundidad.

\txb{Implementaci{\'o}n}:\nl \v{fringe} = cola LIFO , es decir,
coloca los sucesores al frente

\epsfxsize=0.5\textwidth
\fig{\sfile{figures}{dfs-progress03.ps}}


%%%%%%%%%%%% Slide %%%%%%%%%%%%%%%%%%%%%%%%%%%%%%%%%%%%%%%%%%%%%%%%%%%
\heading{B{\'u}squeda primero en profundidad}

Expande el nodo no expandido de major profundidad.

\txb{Implementaci{\'o}n}:\nl \v{fringe} = cola LIFO , es decir,
coloca los sucesores al frente

\epsfxsize=0.5\textwidth
\fig{\sfile{figures}{dfs-progress04.ps}}


%%%%%%%%%%%% Slide %%%%%%%%%%%%%%%%%%%%%%%%%%%%%%%%%%%%%%%%%%%%%%%%%%%
\heading{B{\'u}squeda primero en profundidad}

Expande el nodo no expandido de major profundidad.

\txb{Implementaci{\'o}n}:\nl \v{fringe} = cola LIFO , es decir,
coloca los sucesores al frente

\epsfxsize=0.5\textwidth
\fig{\sfile{figures}{dfs-progress05.ps}}


%%%%%%%%%%%% Slide %%%%%%%%%%%%%%%%%%%%%%%%%%%%%%%%%%%%%%%%%%%%%%%%%%%
\heading{B{\'u}squeda primero en profundidad}

Expande el nodo no expandido de major profundidad.

\txb{Implementaci{\'o}n}:\nl \v{fringe} = cola LIFO , es decir,
coloca los sucesores al frente

\epsfxsize=0.5\textwidth
\fig{\sfile{figures}{dfs-progress06.ps}}


%%%%%%%%%%%% Slide %%%%%%%%%%%%%%%%%%%%%%%%%%%%%%%%%%%%%%%%%%%%%%%%%%%
\heading{B{\'u}squeda primero en profundidad}

Expande el nodo no expandido de major profundidad.

\txb{Implementaci{\'o}n}:\nl \v{fringe} = cola LIFO , es decir,
coloca los sucesores al frente

\epsfxsize=0.5\textwidth
\fig{\sfile{figures}{dfs-progress07.ps}}


%%%%%%%%%%%% Slide %%%%%%%%%%%%%%%%%%%%%%%%%%%%%%%%%%%%%%%%%%%%%%%%%%%
\heading{B{\'u}squeda primero en profundidad}

Expande el nodo no expandido de major profundidad.

\txb{Implementaci{\'o}n}:\nl \v{fringe} = cola LIFO , es decir,
coloca los sucesores al frente

\epsfxsize=0.5\textwidth
\fig{\sfile{figures}{dfs-progress08.ps}}


%%%%%%%%%%%% Slide %%%%%%%%%%%%%%%%%%%%%%%%%%%%%%%%%%%%%%%%%%%%%%%%%%%
\heading{B{\'u}squeda primero en profundidad}

Expande el nodo no expandido de major profundidad.

\txb{Implementaci{\'o}n}:\nl \v{fringe} = cola LIFO , es decir,
coloca los sucesores al frente

\epsfxsize=0.5\textwidth
\fig{\sfile{figures}{dfs-progress09.ps}}


%%%%%%%%%%%% Slide %%%%%%%%%%%%%%%%%%%%%%%%%%%%%%%%%%%%%%%%%%%%%%%%%%%
\heading{B{\'u}squeda primero en profundidad}

Expande el nodo no expandido de major profundidad.

\txb{Implementaci{\'o}n}:\nl \v{fringe} = cola LIFO , es decir,
coloca los sucesores al frente

\epsfxsize=0.5\textwidth
\fig{\sfile{figures}{dfs-progress10.ps}}


%%%%%%%%%%%% Slide %%%%%%%%%%%%%%%%%%%%%%%%%%%%%%%%%%%%%%%%%%%%%%%%%%%
\heading{B{\'u}squeda primero en profundidad}

Expande el nodo no expandido de major profundidad.

\txb{Implementaci{\'o}n}:\nl \v{fringe} = cola LIFO , es decir,
coloca los sucesores al frente

\epsfxsize=0.5\textwidth
\fig{\sfile{figures}{dfs-progress11.ps}}


%%%%%%%%%%%% Slide %%%%%%%%%%%%%%%%%%%%%%%%%%%%%%%%%%%%%%%%%%%%%%%%%%%
\heading{B{\'u}squeda primero en profundidad}

Expande el nodo no expandido de major profundidad.

\txb{Implementaci{\'o}n}:\nl \v{fringe} = cola LIFO , es decir,
coloca los sucesores al frente

\epsfxsize=0.5\textwidth \fig{\sfile{figures}{dfs-progress12.ps}}


%%%%%%%%%%%% Slide %%%%%%%%%%%%%%%%%%%%%%%%%%%%%%%%%%%%%%%%%%%%%%%%%%%
\heading{\small Propiedades de la b{\'u}squeda primero en
profundidad}

\q{Completa}

%%%%%%%%%%%% Slide %%%%%%%%%%%%%%%%%%%%%%%%%%%%%%%%%%%%%%%%%%%%%%%%%%%
\heading{\small Propiedades de la b{\'u}squeda primero en
profundidad}

\q{Completa} No: falla en espacios de estados infinitos y espacios
con ciclos.\nl Se puede modificar para evitar repetir estados a lo
largo de un camino\nnl $\Rightarrow$ lo cual hace que sea completa
en espacios finitos (a{\'u}n con ciclos).

\q{Tiempo}

%%%%%%%%%%%% Slide %%%%%%%%%%%%%%%%%%%%%%%%%%%%%%%%%%%%%%%%%%%%%%%%%%%
\heading{\small Propiedades de la b{\'u}squeda primero en
profundidad}

\q{Completa} No: falla en espacios de estados infinitos y espacios
con ciclos.\nl Se puede modificar para evitar repetir estados a lo
largo de un camino\nnl $\Rightarrow$ lo cual hace que sea completa
en espacios finitos (a{\'u}n con ciclos).

\q{Tiempo} $O(b^m)$: muy malo si $m$ es mucho mayor que $d$\nl
pero si las soluciones son ``densas'' en el espacio, puede ser
mucho m{\'a}s r{\'a}pida que primero en anchura.

\q{Espacio}

%%%%%%%%%%%% Slide %%%%%%%%%%%%%%%%%%%%%%%%%%%%%%%%%%%%%%%%%%%%%%%%%%%
\heading{\small Propiedades de la b{\'u}squeda primero en
profundidad}

\q{Completa} No: falla en espacios de estados infinitos y espacios
con ciclos.\nl Se puede modificar para evitar repetir estados a lo
largo de un camino\nnl $\Rightarrow$ lo cual hace que sea completa
en espacios finitos (a{\'u}n con ciclos).

\q{Tiempo} $O(b^m)$: muy malo si $m$ es mucho mayor que $d$\nl
pero si las soluciones son ``densas'' en el espacio, puede ser
mucho m{\'a}s r{\'a}pida que primero en anchura.

\q{Espacio} $O(bm)$, es decir, lineal en el espacio de estados!

\q{Optima}

%%%%%%%%%%%% Slide %%%%%%%%%%%%%%%%%%%%%%%%%%%%%%%%%%%%%%%%%%%%%%%%%%%
\heading{\small Propiedades de la b{\'u}squeda primero en
profundidad}

\q{Completa} No: falla en espacios de estados infinitos y espacios
con ciclos.\nl Se puede modificar para evitar repetir estados a lo
largo de un camino\nnl $\Rightarrow$ lo cual hace que sea completa
en espacios finitos (a{\'u}n con ciclos).

\q{Tiempo} $O(b^m)$: muy malo si $m$ es mucho mayor que $d$\nl
pero si las soluciones son ``densas'' en el espacio, puede ser
mucho m{\'a}s r{\'a}pida que primero en anchura.

\q{Espacio} $O(bm)$, es decir, lineal en el espacio de estados!

\q{Optima} No


%%%%%%%%%%%% Slide %%%%%%%%%%%%%%%%%%%%%%%%%%%%%%%%%%%%%%%%%%%%%%%%%%%
\heading{B{\'u}squeda de profundidad limitada}

= b{\'u}squeda primero en profundidad con l{\'\i}mite de profundidad $l$,\\
es decir, los nodos a profundidad $l$ no tienen sucesores.

\txb{Implementaci{\'o}n recursiva}:

\input{\file{algorithms}{recursive-dls-algorithm}}

%%%%%%%%%%%% Slide %%%%%%%%%%%%%%%%%%%%%%%%%%%%%%%%%%%%%%%%%%%%%%%%%%%
\heading{B{\'u}squeda de profundizamiento iterativo}

\input{\file{algorithms}{ids-algorithm}}



%%%%%%%%%%%% Slide %%%%%%%%%%%%%%%%%%%%%%%%%%%%%%%%%%%%%%%%%%%%%%%%%%%
\heading{B{\'u}squeda de profundizamiento iterativo $l=0$}

\vspace*{0.3in}

\epsfxsize=1.1\textwidth
\fig{\sfile{figures}{ids-progress1.ps}}

%%%%%%%%%%%% Slide %%%%%%%%%%%%%%%%%%%%%%%%%%%%%%%%%%%%%%%%%%%%%%%%%%%
\heading{B{\'u}squeda de profundizamiento iterativo $l=1$}

\vspace*{0.3in}

\epsfxsize=1.1\textwidth
\fig{\sfile{figures}{ids-progress2.ps}}

%%%%%%%%%%%% Slide %%%%%%%%%%%%%%%%%%%%%%%%%%%%%%%%%%%%%%%%%%%%%%%%%%%
\heading{B{\'u}squeda de profundizamiento iterativo $l=2$}

\vspace*{0.3in}

\epsfxsize=1.1\textwidth
\fig{\sfile{figures}{ids-progress3.ps}}

%%%%%%%%%%%% Slide %%%%%%%%%%%%%%%%%%%%%%%%%%%%%%%%%%%%%%%%%%%%%%%%%%%
\heading{B{\'u}squeda de profundizamiento iterativo $l=3$}

\vspace*{0.3in}

\epsfxsize=1.1\textwidth
\fig{\sfile{figures}{ids-progress4.ps}}




%%%%%%%%%%%% Slide %%%%%%%%%%%%%%%%%%%%%%%%%%%%%%%%%%%%%%%%%%%%%%%%%%%
\heading{\small Propiedades de la b{\'u}squeda de profundizamiento
iterativo}

\q{Completa}

%%%%%%%%%%%% Slide %%%%%%%%%%%%%%%%%%%%%%%%%%%%%%%%%%%%%%%%%%%%%%%%%%%
\heading{\small Propiedades de la b{\'u}squeda de profundizamiento
iterativo}

\q{Completa} S{\'\i}

\q{Tiempo}

%%%%%%%%%%%% Slide %%%%%%%%%%%%%%%%%%%%%%%%%%%%%%%%%%%%%%%%%%%%%%%%%%%
\heading{\small Propiedades de la b{\'u}squeda de profundizamiento
iterativo}

\q{Completa} S{\'\i}

\q{Tiempo} $(d+1)b^0 + d b^1 + (d-1)b^2 + \ldots + b^d = O(b^d)$

\q{Espacio}

%%%%%%%%%%%% Slide %%%%%%%%%%%%%%%%%%%%%%%%%%%%%%%%%%%%%%%%%%%%%%%%%%%
\heading{\small Propiedades de la b{\'u}squeda de profundizamiento
iterativo}

\q{Completa} S{\'\i}

\q{Tiempo} $(d+1)b^0 + d b^1 + (d-1)b^2 + \ldots + b^d = O(b^d)$

\q{Espacio} $O(bd)$

\q{Optima}

%%%%%%%%%%%% Slide %%%%%%%%%%%%%%%%%%%%%%%%%%%%%%%%%%%%%%%%%%%%%%%%%%%
\heading{\small Propiedades de la b{\'u}squeda de profundizamiento
iterativo}

\q{Completa} S{\'\i}

\q{Tiempo} $(d+1)b^0 + d b^1 + (d-1)b^2 + \ldots + b^d = O(b^d)$

\q{Espacio} $O(bd)$

\q{Optima} S{\'\i}, si el costo del paso es = 1\nl Puede ser
modificada para que explore el {\'a}rbol de costo uniforme.

Comparaci{\'o}n num{\'e}rica para $b=10$ y $d=5$. La soluci{\'o}n
a la derecha extrema:
\begin{eqnarray*}
N(\mbox{Prof. iterativo}) &=& 50 + 400 + 3,000 + 20,000 + 100,000 = 123,450 \\
N(\mbox{Primero anchura}) &=& 10 + 100 + 1,000 + 10,000 + 100,000
+ 999,990 = 1,111,100
\end{eqnarray*}



%%%%%%%%%%%% Slide %%%%%%%%%%%%%%%%%%%%%%%%%%%%%%%%%%%%%%%%%%%%%%%%%%%
\heading{Resumen de los algoritmos}

\input{\file{tables}{search-summary-table}}


%%%%%%%%%%%% Slide %%%%%%%%%%%%%%%%%%%%%%%%%%%%%%%%%%%%%%%%%%%%%%%%%%%
\heading{El super problema con los estados repetidos}


La incapacidad de detectar los estados repetidos puede convertir
un problema lineal en uno exponencial!.

\vspace*{0.3in}

\epsfxsize=0.7\maxfigwidth
\fig{\file{figures}{ribbon-space.ps}}


%%%%%%%%%%%% Slide %%%%%%%%%%%%%%%%%%%%%%%%%%%%%%%%%%%%%%%%%%%%%%%%%%%
\heading{B{\'u}squeda en grafos}

\input{\file{algorithms}{graph-search-algorithm}}



%%%%%%%%%%%% Slide %%%%%%%%%%%%%%%%%%%%%%%%%%%%%%%%%%%%%%%%%%%%%%%%%%%
\heading{Resumen}

La formulaci{\'o}n de un problema requiere usualmente que el
agente abstraiga detalles de la realidad tal como se le presenta,
de manera que pueda definir un espacio de estados factible para la
b{\'u}squeda.

Hay una variedad de estrategias de b{\'u}squeda no informadas.

La t{\'e}cnica de profundizamiento iterativo usa solamente espacio
lineal\\ y no mucho m{\'a}s tiempo que los otros algoritmos no
informados.

Volvamos con la formulaci{\'o}n del problema. ?`Qu{\'e} nos falta?
?` Qu{\'e} otra clase de conocimiento se puede usar en las
b{\'u}squedas?

\end{huge}
\end{document}
