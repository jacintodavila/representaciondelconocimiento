\documentclass{article}
\usepackage{fleqn}
\usepackage{epsf}
\usepackage[dvips]{color}
\usepackage{aima2e-slides}
%\usepackage[landscape]{geometry}

\begin{document}

\begin{huge}
\titleslide{Agentes Inteligentes}{Unidad 2}

\sf

%%%%%%%%%%%% Slide %%%%%%%%%%%%%%%%%%%%%%%%%%%%%%%%%%%%%%%%%%%%%%%%%%%
\heading{Intelijencia}

Intelijencia, dame\\
el nombre exacto de las cosas!\\
$\ldots$ Que mi palabra sea\\
la cosa misma\\
creada por mi alma nuevamente.\\
Que por m{\'\i} vayan todos\\
los que no las conocen, a las cosas;\\
que por m{\'\i} vayan todos\\
los que ya las olvidan, a las cosas;\\
que por m{\'\i} vayan todos\\
los mismos que las aman, a las cosas$\ldots$\\
!`Intelijencia, dame\\
el nombre exacto, y tuyo,\\
y suyo, y m{\'\i}o, de las cosas!

            Juan Ram{\'o}n Jim{\'e}nez.

%%%%%%%%%%%% Slide %%%%%%%%%%%%%%%%%%%%%%%%%%%%%%%%%%%%%%%%%%%%%%%%%%%
\heading{Outline}

\blob Los agentes y sus ambientes.

\blob Racionalidad

\blob Medidas de rendimiento, el ambiente, los actuadores y los
sensores, PEAS (Performance measure, Environment, Actuators,
Sensors)

\blob Tipos de ambientes

\blob Tipos de agentes

%%%%%%%%%%%% Slide %%%%%%%%%%%%%%%%%%%%%%%%%%%%%%%%%%%%%%%%%%%%%%%%%%%
\heading{Los agentes y sus ambientes}

\epsfxsize=0.65\textwidth
\fig{\file{figures}{agent-environment.ps}}

Los \txr{Agentes} pueden ser humanos, robots, softbots,
dispositivos como el termostato y muchos otros.

La \txr{funci{\'o}n agente} transforma historias de ``perceptos"
en acciones:
\[f: {\cal P}^* \rightarrow {\cal A}\]
El \txr{programa agente} se ejecuta sobre una arquitectura
f{\'\i}sica particular para ``producir" $f$


%%%%%%%%%%%% Slide %%%%%%%%%%%%%%%%%%%%%%%%%%%%%%%%%%%%%%%%%%%%%%%%%%%
\heading{El mundo de la aspiradora}

\vspace*{0.3in}

\epsfxsize=0.65\textwidth
\fig{\file{figures}{vacuum2-environment.ps}}

Perceptos: ubicaci{\'o}n y contenidos, e.g., $[A,Dirty]$

Acciones: $left$, $right$, $suck$, $noOp$

%%%%%%%%%%%% Slide %%%%%%%%%%%%%%%%%%%%%%%%%%%%%%%%%%%%%%%%%%%%%%%%%%%
\heading{Un agente aspiradora}

\input{tables/vacuum-agent-function-table}%

\medskip

\input{algorithms/reflex-vacuum-agent-algorithm}

?`Qu{\'e} es la funci{\'o}n \txg{Right}? \\
?`Podemos implementarla con un peque{\~n}o programa agente?


%%%%%%%%%%%% Slide %%%%%%%%%%%%%%%%%%%%%%%%%%%%%%%%%%%%%%%%%%%%%%%%%%%
\heading{Racionalidad}


Una \txr{medida fija de rendimiento} evalua una \txg{secuencia de
ambientes}\al

-- ?`un punto por cuadro limpio en un tiempo $T$?\al -- ?`un punto
por cuadro limpio por unidad de tiempo, menos uno por
movimiento?\al -- ?`Penalizaci{\'o}n por ${}> k$ cuadro sucios?

Un \txr{agente racional} escoge aquella acci{\'o}n que maximiza el
valor \txr{esperado} de su medida de rendimiento \txr{respecto a
cierta secuencia de perceptos}

Racional $\neq$ omnisciente\\
Racional $\neq$ clarividente\\
Racional $\neq$ exitoso

Racional $\implies$ exploraci{\'o}n, aprendizaje, autonom{\'\i}a.


%%%%%%%%%%%% Slide %%%%%%%%%%%%%%%%%%%%%%%%%%%%%%%%%%%%%%%%%%%%%%%%%%%
\heading{PEAS}

Para dise{\~n}ar un agente racional, debemos especificar el
ambiente de trabajo.

Considere, por ejemplo, el dise{\~n}o de un taxi autom{\'a}tico:


\q{Medidas de rendimiento}

\q{Ambiente}

\q{Actuadores}

\q{Sensores}


%%%%%%%%%%%% Slide %%%%%%%%%%%%%%%%%%%%%%%%%%%%%%%%%%%%%%%%%%%%%%%%%%%
\heading{PEAS}

Para dise{\~n}ar un agente racional, debemos especificar el
ambiente de trabajo.

Considere, por ejemplo, el dise{\~n}o de un taxi autom{\'a}tico:


\q{Medidas de rendimiento} seguridad, destinos posibles,
ganancias, status legal, confort, $\ldots$

\q{Ambiente} calles y avenidas, tr{\'a}fico, peatones, clima,
$\ldots$

\q{Actuadores} volante, acelerador, freno, bocina, corneta,
tablero, $\ldots$

\q{Sensores} video, veloc{\'\i}metro, va�lvulas, sensores del
motor, teclados, GPS, $\ldots$.


%%%%%%%%%%%% Slide %%%%%%%%%%%%%%%%%%%%%%%%%%%%%%%%%%%%%%%%%%%%%%%%%%%
\heading{En agente comprador en Internet}

\q{Medidas de rendimiento}

\q{Ambiente}

\q{Actuadores}

\q{Sensores}

%%%%%%%%%%%% Slide %%%%%%%%%%%%%%%%%%%%%%%%%%%%%%%%%%%%%%%%%%%%%%%%%%%
\heading{Tipos de ambientes}

\begin{mytabular}{@{\extracolsep\fill}|@{\squad}l@{\quad}|cccc@{\squad}|}
\hline
\tabhead & {Solitario} & {Backgammon} & {Compras en Internet} & {Taxi} \\
\hline
\tabtop
\q{Observable}   &   &   &   &  \\
\q{Determ{\'\i}nistico} &   &   &   &  \\
\q{Epis{\'o}dico}      &   &   &   &  \\
\q{Est{\'a}tico}       &   &   &   &  \\
\q{Discreto}       &   &   &   &  \\
\tabbot
\q{Un-solo-agente}     &   &   &   &  \\
\hline
\end{mytabular}


%%%%%%%%%%%% Slide %%%%%%%%%%%%%%%%%%%%%%%%%%%%%%%%%%%%%%%%%%%%%%%%%%%
\heading{Tipos de ambientes}

\begin{mytabular}{@{\extracolsep\fill}|@{\squad}l@{\quad}|cccc@{\squad}|}
\hline
\tabhead & {Solitario} & {Backgammon} & {Compras en Internet} & {Taxi} \\
\hline \tabtop
\q{Observable}   &  Si &  Si &  No &  No \\
\q{Determ{\'\i}nistico} &   &   &   &  \\
\q{Epis{\'o}dico}      &   &   &   &  \\
\q{Est{\'a}tico}       &   &   &   &  \\
\q{Discreto}       &   &   &   &  \\
\tabbot
\q{Un-solo-agente}     &   &   &   &  \\
\hline
\end{mytabular}


%%%%%%%%%%%% Slide %%%%%%%%%%%%%%%%%%%%%%%%%%%%%%%%%%%%%%%%%%%%%%%%%%%
\heading{Tipos de ambiente}

\begin{mytabular}{@{\extracolsep\fill}|@{\squad}l@{\quad}|cccc@{\squad}|}
\hline
\tabhead & {Solitario} & {Backgammon} & {Compras en Internet} & {Taxi} \\
\hline \tabtop
\q{Observable}   &  Si &  Si &  No &  No \\
\q{Determ{\'\i}nistico} & Si  & No  &  Parcialmente & No \\
\q{Epis{\'o}dico}      &   &   &   &  \\
\q{Est{\'a}tico}       &   &   &   &  \\
\q{Discreto}       &   &   &   &  \\
\tabbot
\q{Un-solo-agente}     &   &   &   &  \\
\hline
\end{mytabular}

%%%%%%%%%%%% Slide %%%%%%%%%%%%%%%%%%%%%%%%%%%%%%%%%%%%%%%%%%%%%%%%%%%
\heading{Tipos de ambientes}

\begin{mytabular}{@{\extracolsep\fill}|@{\squad}l@{\quad}|cccc@{\squad}|}
\hline
\tabhead & {Solitario} & {Backgammon} & {Compras en Internet} & {Taxi} \\
\hline \tabtop
\q{Observable}   &  Si &  Si &  No &  No \\
\q{Determ{\'\i}nistico} & Si  & No  &  Parcialmente & No \\
\q{Epis{\'o}dico}      & No  & No  &  No & No \\
\q{Est{\'a}tico}       &   &   &   &  \\
\q{Discreto}       &   &   &   &  \\
\tabbot
\q{Un-solo-agente}     &   &   &   &  \\
\hline
\end{mytabular}


%%%%%%%%%%%% Slide %%%%%%%%%%%%%%%%%%%%%%%%%%%%%%%%%%%%%%%%%%%%%%%%%%%
\heading{Tipos de ambientes}

\begin{mytabular}{@{\extracolsep\fill}|@{\squad}l@{\quad}|cccc@{\squad}|}
\hline
\tabhead & {Solitario} & {Backgammon} & {Compras en Internet} & {Taxi} \\
\hline \tabtop
\q{Observable}   &  Si &  Si &  No &  No \\
\q{Determ{\'\i}nistico} & Si  & No  &  Parcialmente & No \\
\q{Epis{\'o}dico}      & No  & No  &  No & No \\
\q{Est{\'a}tico}       &  Si &  Semi  & Semi  &  No \\
\q{Discreto}       &   &   &   &  \\
\tabbot
\q{Un-solo-agente}     &   &   &   &  \\
\hline
\end{mytabular}

%%%%%%%%%%%% Slide %%%%%%%%%%%%%%%%%%%%%%%%%%%%%%%%%%%%%%%%%%%%%%%%%%%
\heading{Tipos de ambientes}

\begin{mytabular}{@{\extracolsep\fill}|@{\squad}l@{\quad}|cccc@{\squad}|}
\hline
\tabhead & {Solitario} & {Backgammon} & {Compras en Internet} & {Taxi} \\
\hline \tabtop
\q{Observable}   &  Si &  Si &  No &  No \\
\q{Determ{\'\i}nistico} & Si  & No  &  Parcialmente & No \\
\q{Epis{\'o}dico}      & No  & No  &  No & No \\
\q{Est{\'a}tico}       &  Si &  Semi  & Semi  &  No \\
\q{Discreto}       & Si  & Si  & Si  & No \\
\tabbot
\q{Un-solo-agente}     &   &   &   &  \\
\hline
\end{mytabular}

%%%%%%%%%%%% Slide %%%%%%%%%%%%%%%%%%%%%%%%%%%%%%%%%%%%%%%%%%%%%%%%%%%
\heading{Tipos de ambientes}

\begin{mytabular}{@{\extracolsep\fill}|@{\squad}l@{\quad}|cccc@{\squad}|}
\hline
\tabhead & {Solitario} & {Backgammon} & {Compras en Internet} & {Taxi} \\
\hline \tabtop
\q{Observable}   &  Si &  Si &  No &  No \\
\q{Determ{\'\i}nistico} & Si  & No  &  Parcialmente & No \\
\q{Epis{\'o}dico}      & No  & No  &  No & No \\
\q{Est{\'a}tico}       &  Si &  Semi  & Semi  &  No \\
\q{Discreto}       & Si  & Si  & Si  & No \\
\tabbot
\q{Un-solo-agente}     & Si  & No  &  Si (salvo subastas) & No \\
\hline
\end{mytabular}

El tipo de ambiente determina, en buena medida, el dise{\~n}o del
agente.

El mundo real es (?`desde luego?) parcialmente observable,
estoc{\'a}stico, secuencial, din{\'a}mico, continuo y multiagente.

%%%%%%%%%%%% Slide %%%%%%%%%%%%%%%%%%%%%%%%%%%%%%%%%%%%%%%%%%%%%%%%%%%
\heading{Tipos de agentes}

Listamos 4 tipos b{\'a}sicos en orden de generalidad creciente
(segu'n R\&N):\al

-- Agente simple reflejo\al

-- Agente reflejo con estado\al

-- Agentes orientado a metas\al

-- Agentes basados en utilidad

Todos se pueden convertir en agentes aprendices.

Ver tambi{\'e}n la jer{\'a}rquia de [Davila, Uzc{\'a}tegui,
Tucci].

%%%%%%%%%%%% Slide %%%%%%%%%%%%%%%%%%%%%%%%%%%%%%%%%%%%%%%%%%%%%%%%%%%
\heading{Agente simple reflejo}

\epsfxsize=0.95\textwidth
\fig{\file{figures}{simple-reflex-agent.ps}}


%%%%%%%%%%%% Slide %%%%%%%%%%%%%%%%%%%%%%%%%%%%%%%%%%%%%%%%%%%%%%%%%%%
\heading{Agente reflejo con estado}

\epsfxsize=0.95\textwidth
\fig{\file{figures}{reflex+state-agent.ps}}


%%%%%%%%%%%% Slide %%%%%%%%%%%%%%%%%%%%%%%%%%%%%%%%%%%%%%%%%%%%%%%%%%%
\heading{Agente orientado a metas}

\epsfxsize=0.95\textwidth
\fig{\file{figures}{goal-based-agent.ps}}


%%%%%%%%%%%% Slide %%%%%%%%%%%%%%%%%%%%%%%%%%%%%%%%%%%%%%%%%%%%%%%%%%%
\heading{Agente basado en utilidad}

\epsfxsize=0.95\textwidth
\fig{\file{figures}{utility-based-agent.ps}}

%%%%%%%%%%%% Slide %%%%%%%%%%%%%%%%%%%%%%%%%%%%%%%%%%%%%%%%%%%%%%%%%%%
\heading{Agente aprendiz}

\epsfxsize=0.95\textwidth
\fig{\file{figures}{learning-agent.ps}}


%%%%%%%%%%%% Slide %%%%%%%%%%%%%%%%%%%%%%%%%%%%%%%%%%%%%%%%%%%%%%%%%%%
\heading{c{\'o}digo AIMA}

\begin{verbatim}
(setq joe (make-agent :name 'joe :body (make-agent-body)
                      :program (make-dumb-agent-program)))

(defun make-dumb-agent-program ()
  (let ((memory nil))
    #'(lambda (percept)
        (push percept memory)
        'no-op)))
\end{verbatim}


%%%%%%%%%%%% Slide %%%%%%%%%%%%%%%%%%%%%%%%%%%%%%%%%%%%%%%%%%%%%%%%%%%
\heading{c{\'o}digo Bioinformantes: un tutor}

\begin{verbatim}
% Integrity constraints.
if biotutor_requested, not(class_opened) then open_class. \\
if class_opened, not(seen_something) then show_page_1.\\
if class_opened, seen_page_1, not(pending_queries) then
close_class.\\
if seen_page_1, want_a_demo then do_method.\\
if user_continue, method_running then do_yes.\\
if question_asked then answer_question.\\

% Definitions
to seen_something do seen_page_1.\\
to seen_something do user_continue.
\end{verbatim}

%%%%%%%%%%%% Slide %%%%%%%%%%%%%%%%%%%%%%%%%%%%%%%%%%%%%%%%%%%%%%%%%%%
\heading{c{\'o}digo Galatea: Un colono}

\begin{verbatim}
executable(invade). executable(tala). executable(cultiva).
executable(invade_vecino). executable(anexa).
executable(afiliate). executable(vende_tierra).
executable(subsiste). executable(compra_tierra).
executable(siembra_ganado).

observable(tienes_tierra). observable(tienes_dinero).
observable(extension_posible).
observable(te_solicitan_apoyo_politico).

si not(tienes_tierra), not(tienes_dinero) entonces
invade_y_subsiste.\\
si tienes_tierra, extension_posible entonces invade_vecino.\\
si te_solicitan_apoyo_politico, te_dan_tierra entonces afiliate.
si tienes_tierra, not(tienes_dinero) entonces
vende_y_subsiste_con_eso.\\
si tienes_tierra, tienes_dinero entonces explotacion_ganadera.

para invade_y_subsiste haga
    invade, % define un terreno para este agente.
    tala,   % inicia una trayectoria de tala en ese terreno,
            % lineal, 20 hectareas en 5 a#os.
    cultiva.    % inicia una trayectoria de cultivo en ese terreno.

para invade_vecino haga
    anexa.  % inicia una trayectoria de tala o extendiende
        % la del resto de su propiedad
        % extendiende la trayectoria del capital.

para vende_y_subsiste_con_eso haga
    vende_tierra, % Inicia trayectoria de capital con intereses.
    subsiste. % Inicia una trayectoria de gastos de supervivencia.

para explotacion_ganadera haga
    compra_tierra,  % afecta trayectoria del capital.
    siembra_ganado. % inicia trayectoria del reba#o y del capital.
\end{verbatim}

%%%%%%%%%%%% Slide %%%%%%%%%%%%%%%%%%%%%%%%%%%%%%%%%%%%%%%%%%%%%%%%%%%
\heading{c{\'o}digo Galatea: Un politiquero}

\begin{verbatim}
executable(solicita_apoyo). executable(ofrece_tierra).

observable(tienes_tierra). observable(observo_colono).

% Integrity constraints.
si observo_colono, tienes_tierra entonces negocia_tierra_por_voto.

% Definitions
para negocia_tierra_por_voto haga
    solicita_apoyo,
    ofrece_tierra.

observe tienes_tierra. observe observo_colono.
\end{verbatim}

\end{huge}
\end{document}
